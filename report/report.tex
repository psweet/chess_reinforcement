\documentclass[12pt]{article}
\title{Phase 1: Building a Chess Player with Reinforcement Learning}
\author{Carlos Puerta}

\begin{document}

\maketitle

\section{Environment Setup}
The intial environment setup was made using both the `python-chess' library, as well as the `gymnasium' library. Gymnasium seemed to be the most commonly used reinforcement learning library, so I created an environement based on their implementation of what methods and features an environment should have.

\subsection{Software Specs}
I am using Python 3.11.2, with the latest version of both python-chess as well as gymnasium. 

\subsection{Methods}
As of right now, I have implemented the methods specified in the gymnasium documentation, namely, step, reset, and render. 

The methods perform the actions that you can assume from their name: 
\begin{enumerate}
    \item Step advances the game
    \item Reset restarts the game
    \item Render shows the game
\end{enumerate}

Step takes an action to advance the game. Using this set up, I am able to feed any desired action into the step function to advance the simulation. At the moment, these actions are randomly taken from the list of legal actions.

Using this set up, I am able to simulate complete games very quickly.

\end{document}